\documentclass[10pt,ngerman,dvipsnames,ignorenonframetext,]{beamer}
\setbeamertemplate{caption}[numbered]
\setbeamertemplate{caption label separator}{: }
\setbeamercolor{caption name}{fg=normal text.fg}
\beamertemplatenavigationsymbolsempty
\usepackage{lmodern}
\usepackage{amssymb,amsmath}
\usepackage{ifxetex,ifluatex}
\usepackage{fixltx2e} % provides \textsubscript
\ifnum 0\ifxetex 1\fi\ifluatex 1\fi=0 % if pdftex
  \usepackage[T1]{fontenc}
  \usepackage[utf8]{inputenc}
\else % if luatex or xelatex
  \ifxetex
    \usepackage{mathspec}
  \else
    \usepackage{fontspec}
  \fi
  \defaultfontfeatures{Ligatures=TeX,Scale=MatchLowercase}
\fi
\usetheme[]{NPBT}
\usecolortheme{FOM_ifes}
\useoutertheme{FOM_ifes}
% use upquote if available, for straight quotes in verbatim environments
\IfFileExists{upquote.sty}{\usepackage{upquote}}{}
% use microtype if available
\IfFileExists{microtype.sty}{%
\usepackage{microtype}
\UseMicrotypeSet[protrusion]{basicmath} % disable protrusion for tt fonts
}{}
\ifnum 0\ifxetex 1\fi\ifluatex 1\fi=0 % if pdftex
  \usepackage[shorthands=off,main=ngerman]{babel}
\else
  \usepackage{polyglossia}
  \setmainlanguage[]{german}
\fi
\newif\ifbibliography
\hypersetup{
            pdftitle={Sommerumfrage 2017 -- Wahlverhalten der Deutschen 2017},
            pdfauthor={Prof.~Dr.~Oliver Gansser},
            colorlinks=true,
            linkcolor=blue,
            citecolor=Blue,
            urlcolor=blue,
            breaklinks=true}
\urlstyle{same}  % don't use monospace font for urls

% Prevent slide breaks in the middle of a paragraph:
\widowpenalties 1 10000
\raggedbottom

\AtBeginPart{
  \let\insertpartnumber\relax
  \let\partname\relax
  \frame{\partpage}
}
\AtBeginSection{
  \ifbibliography
  \else
    \let\insertsectionnumber\relax
    \let\sectionname\relax
    \frame{\sectionpage}
  \fi
}
\AtBeginSubsection{
  \let\insertsubsectionnumber\relax
  \let\subsectionname\relax
  \frame{\subsectionpage}
}

\setlength{\parindent}{0pt}
\setlength{\parskip}{6pt plus 2pt minus 1pt}
\setlength{\emergencystretch}{3em}  % prevent overfull lines
\providecommand{\tightlist}{%
  \setlength{\itemsep}{0pt}\setlength{\parskip}{0pt}}
\setcounter{secnumdepth}{0}
\AtBeginSection[]{\relax}

\title{Sommerumfrage 2017 -- Wahlverhalten der Deutschen 2017}
\subtitle{Ergebnisdiagramme}
\author{Prof.~Dr.~Oliver Gansser}
\institute{FOM}
\date{WS17}

\begin{document}
\frame{\titlepage}

\section{Factsheet}\label{factsheet}

\mode<all> \setinversetitle
\mode*

\begin{frame}{Sommerumfrage 2017}

\textbf{Zielsetzung des Umfrageprojekts:}

Untersuchung der wichtigen Themen für die Bundestagswahl

\textbf{Feldzugang:}

``Face-to-face-Interviews'' mit standardisiertem Fragebogen

\textbf{Zielgruppe/Teilnehmer:}

Auskunftspersonen im Alter ab 16 Jahren, quotiert nach Alter und
Geschlecht, Verteilung der Quotenmerkmale gemäß der
Bevölkerungsvorausberechnung des Statistischen Bundesamtes für 2017.

\textbf{Dauer der Feldphase:}

01.03.2017 bis 14.05.2017

\textbf{Wissenschaftliche Leitung:}

Prof.~Dr.~Oliver Gansser

\textbf{Wissenschaftliche Mitarbeit:}

Dipl.-Hdl. Christina Reich

\textbf{Anzahl der ausgewerteten Interviews:}

n = 20.895

\begin{block}{}

Die Darstellungen sind ausschließlich als Diskussionsgrundlage
konzipiert und sind ohne die mündlichen Erläuterungen zur Präsentation
unvollständig. Diese Unterlage kann daher nur im Zusammenhang mit einer
Präsentation gesehen werden.

\end{block}

\end{frame}

\section{Sommerumfrage}\label{sommerumfrage}

\mode<all> \setnormaltitle
\mode*

\begin{frame}{Befragte Personen nach Alter und Geschlecht}

\end{frame}

\section{}\label{section-1}

\mode<all> \setinversetitle
\mode*

\begin{frame}{Kontakt}

\end{frame}

\end{document}
